\chapter{Background Works}
The background works involved in this thesis in order to create the Smart Home Lab website divided into two parts. The first part is the analysis of selected Smart Home Lab websites of other institutions. The second part is the analysis of the WordPress plugin in order to create the Info-Terminal.

\section{Website Analysis}
In this work, altogether six websites have been analyzed. These include FZI Forschungszentrum, KIT iZEUS, IoTLab Reutlingen, Duke Smart Home Program, MIT Mobile, and MIT Smart Living. In general, criteria such as the layout, contents, main navigation menu, media supported, responsiveness and languages supported have been analyzed.

\subsection{FZI Forschungszentrum}

\subsubsection*{Website Information}
\begin{itemize}
\item URL: https://www.fzi.de/en/home/
\item Web server: Apache
\item Application server: PHP 5.4.45
\item CMS: Typo3
\end{itemize}

\subsubsection*{Web Layout}
The website uses normal layout, not screen-wide. The theme look modern, but the normally used one. The Header contains institution logo on the right, and the top part has option to change language, link to contact page and search bar. Lower part of the header has main navigation menu (Home, News, Research, Our Offer, Work For Us, and About Us). The header is fixed, but collapse when user scroll down.

Body is two column sized. In all pages except home page, the small left column has sub menu respective to that individual page and large right column has the content. On the main page, the the left column is relatively bigger which has slideshow and latest news feed. Right column is relatively small and has list of upcoming events and quicklinks.

Footer is small sized. It has sub menu on the left (Home, Privacy, Legal Notice, Sitemap, Search) It also has social media plugins on the (Xing), \ac{rss} feeds and contact links.

\subsubsection*{Contents}
Contents presented on the website are the home page (featured article/news, upcoming events, latest news, quicklinks), newsfeed, research (research sector, research focuses, projects etc), offers, work for us (carrier page), about us, privacy, legal notice, and sitemap. As analyzed, the website has a lot of contents, perhaps it could have been organized in a better way.

\subsubsection*{Menu}
There are 3 types of menu:
\begin{itemize}
\item Main navigation menu situated at the header (Home, News, Research, Our Offer, Work For Us, and About Us)
\item Sub menu respective to individual pages which is different from one another located at right column of the body part
\item Footer menu located the footer (Home, Privacy, Legal Notice, Sitemap, Search).
\end{itemize}

\subsubsection*{Media}
The website supports the following medias, such as images, slide show, videos, contact form, search bar, social media button.

\subsubsection*{Responsiveness}
The web page is responsive. The header is compressed. The main navigation menu is shifted to the bottom of the page with a link to the menu located at the header. The two column body part is changed to single column on window resize. The sub menu on the individual pages are merged with main navigation menu in hierarchical order, which is situated at the bottom of the page.

\subsubsection*{Languages}
The website supports German and English languages.

\subsection{KIT iZEUS}
\subsubsection*{Website Information}
\begin{itemize}
\item Url: http://www.izeus.kit.edu/62.php
\item Web server: Apache 2.4.10
\item Application server: PHP
\item Server: Debian
\end{itemize}

\subsubsection*{Web Layout}
The web layout is classic normal one. It is not wide and not stretched. The institution logo is located in the header at the top. The sub-menu can be spotted small in the upper right part of the header. The main frame is divided into three columns. The left column contain the main navigation menu. The center column serves as the main content. Lastly, the right column fixed banner. The footer is small and has copyright statement.

The web layout is classic without fancy popup animations, graphics, slider etc. It has to also noted that it is most common and user friendly where most users know how to navigate.

\subsubsection*{Contents}
Contents presented on the website are the home page, energy smart home lab, information materials, project consortium (partners and KIT chairs), publications, press review, links, contact, legals and sitemap. The content look organized, clear to be viewed and read. Navigating through the contents is also easy, but it appears not to be updated anymore.

\subsubsection*{Menu}
The website contains two menus:
\begin{itemize}
\item Small top right menu (Home, Lang Pref, Legals, Sitemap, Link to KIT)
\item Main Menu on Left Column (Home, Energy Smart Home Lab, Information Material, Project Consortium, Publications, Press Review, Links, Contact)
\end{itemize}

Analysis: Both navigation are visible on all pages, static, easy to use, simple hovering effect, highlight on active link.

\subsubsection*{Media}
The website supports the following media, video, images, banners, PDFs, and external links. However, no animation, picture gallery and image slider can be spotted.

\subsubsection*{Responsiveness}
The website is not responsive. It is best viewed with desktop and laptop. Having said that, it is not mobile- or tablet-friendly and not recommended for Microsoft Surface Hub.

\subsubsection*{Languages}
The website has German as the primary language and supports English as an alternative language. The website is designed will on both the languages.

\subsection{IoTLab Reutlingen}
\subsubsection*{Website Information}
\begin{itemize}
\item URL: http://iotlab.reutlingen-university.de
\item Web server: Apache
\item Application server: PHP 5.5.36
\item CMS: Joomla!
\end{itemize}

\subsubsection*{Web Layout}
The website uses a modern web layout and the web page is full-width stretched. The header contains brand logo and horizontal main navigation menu. The header also contains toggle button to open / close main navigation menu. It is fixed and the both menus are duplicated.

The website is created using Joomla! content management system. The home page is single columned with newsfeed. Whereby, the other pages are double column, with right column containing sub-menu and left column showing the main content. Other that that, fixed social media icons (e.g. facebook and twitter) can be found on right edge of the website.

The footer contains copyright statement, impressum link, link to university homepage, link and info about CMS, and link / info about CMS theme. The web layout doesn't look systematic and well organized even though 
it uses modern web templating. The users might face difficulty to get  information or contents they are looking for.

\subsubsection*{Contents}
The website contains the contents such as new feeds on home page, master projects listing (current and finished master projects, team , introduction), publications, thesis, blogs (containing 3 categories: IoTLab-Blog, Mobile Computing and Distributed), systems, contact form, and commenting for blog article.

\subsubsection*{Menu}
The website has three menus:
\begin{itemize}
\item One menu located on the web page header (Home, Master Project, Publications,  Thesis, Blog), but Hidden on mobile / tablet mode.
\item Sidebar Menu which has to be toggle opened/closed using button (Duplicate of  the main menu located on the header with additional menu "Intern")
\item Menu located at left column of pages (Duplicate of the menu with some  additional menus such as Cooperation, News, Events etc.)
\end{itemize}

The menus are duplicated which is inconvenient and confusing for the user to navigate around.

\subsubsection*{Media}
The website contains images, slide shows, and video. It also supports PDFs, banner and animation.

\subsubsection*{Responsiveness}
The website is responsive and adapts to various screen sizes. The contents are readable and viewable using different devices. It is mobile- and tablet-friendly.

\subsubsection*{Languages}
The website supports only English language.

\subsection{Duke Smart Home Program}
\subsubsection*{Website Information}
\begin{itemize}
\item URL: https://smarthome.duke.edu/
\item Web server: Apache 2.4.7
\item Application server: PHP 5.5.9-1
\item CMS: Drupal 4.7
\item Server: Ubuntu 4.20
\end{itemize}

\subsubsection*{Web Layout}
The website uses modern yet simple web layout. The web page is fill-width stretched. The header is divided into three layers. The first layer is smallest and contains two external links. The second layer big and has brand logo and search bar. And the third layer is smaller and has main navigation menu.

The main view of website is divided into two columns. The left column is the main column containing contents. The right column is small and has sub-menu respective to each pages. The footer part is medium sized contains copyright statement, institution logo and a link to contact us.

For analysis, in overall web design looks fine, but unable to scroll through the page using scroll wheel.

\subsubsection*{Contents}
The website consists of following pages. The home page contains picture, about the program, links targeted to different audience, 3 recent research projects. Other that that, there are about, research, out smart dorm, industry, students and contact us pages. The contents are well presented and organized.

\subsubsection*{Menu}
Two types of menu:
\begin{itemize}
\item Main navigation menu located at the header of the website (About, research, our smart dorm, industry, students, contact us)
\item Sub-menu different for individual pages located at right column next to the main layout.
\end{itemize}

The menu are well located and easy for the user to navigate.

\subsubsection*{Media}
The website has images, video, map and banner. However, no gallery, slider and contact form can be spotted.

\subsubsection*{Responsiveness}
The website is fully responsiveness and mobile-friendly. The main navigation menu appears as a left drawer which can be toggled opened and closed.

\subsubsection*{Language}
English is the primary and the only language offered by this website.

\subsection{MIT Mobile Experience Laboratory}
\subsubsection*{Website Information}
\begin{itemize}
\item URL: http://mobile.mit.edu
\item Web server: Apache 2.2.22
\item Application server: PHP 5.4.45
\item CMS: WordPress 4.3.1
\item Theme: WordPress Construct (MEL Edition)
\item Server: Debian 7u8
\end{itemize}

\subsubsection*{Web Layout}
The website use a modern web layout. It is wide screen but not stretched. The contents are arranged in tile-based design. The header contain the institution logo with home page link. The main navigation menu can be found on the header as well.

The main layout is single columned. It has breadcrumb at the top part of the main layout. The projects and event+classes pages have filtering option on the left side. The main navigation menu in the footer is a duplicate from the header' menu. The footer also contains copyright statement, institution and department logo containing external links to their website respectively.

The look of the website is different than normal website as all the contents are arranged in the tiled format.

\subsubsection*{Contents}
The main page contains featured news, upcoming recents+classes, short introduction and contact us section. Other than that, the pages as such, about us, projects that can be filtered according to topic, events+classes that can be filtered according to month/year, team members and sponsors can be found on this website.

\subsubsection*{Menu}
The are two menus on this website:
\begin{itemize}
\item Main navigation menu located at the header (Home, about, projects, events+classes, team, sponsor)
\item Footer menu, same as the main navigation menu
\end{itemize}

The menu looks simple, decent, easy to navigate and direct forward.

\subsubsection*{Media}
The website supports all normal media files. Additionally, slide show, images that can be zoomed-in, breadcrumb, and project has 'related projects' link.

\subsubsection*{Responsiveness}
The website is not responsive as tested. It is not scaling to fit the size of mobile and tablet screens. The user as to browse the website in desktop mode in the mobile devices.

\subsubsection*{Languages}
The website supports only English as primary and the only language.

\subsection{Avea Smarthomes Re:Thinking}
\subsubsection*{Website Information}
\begin{itemize}
\item URL: http://mobile.mit.edu/smartliving/
\item Web server: Apache 2.2.22
\item Application server: PHP 5.4.45
\item CMS: WordPress 4.3.1
\item Theme: WordPress Time (Smart living edit)
\item Server: debian 7u8
\end{itemize}

\subsubsection*{Web Layout}
The web layout of this website looks modern. It is wide but not stretched. The header has two section. First, a broad section for brand logo centralized. Second part has the main navigation menu. The main layout of website is single columned with simple layout. The footer contains copyright statement centralized.

The website has a simple web interface without much fancy items.

\subsubsection*{Contents}
The website has the following pages, home page with slide show, project brief containing project introduction, methodology and projects listing (FLOC, SOL-CHARGE, BOX OF HOLDINGS, COOKBOX.NET) and team.

\subsubsection*{Menu}
There is only one main navigation menu at the header:

\subsubsection*{Media}
The website contains media such as images, slide show and video.

\subsubsection*{Responsiveness}
The website is responsive, but some of the elements could be positioned well (e.g. header). The main navigation menu becomes as a drop down list in the tablet and mobile mode. Since the web page is single column, there is not much change in the body part.

\subsubsection*{Languages}
The website supports only English language.


\section{WordPress Plugins Analysis}
The Info-Terminal has been created with a WordPress plugin. Among many available plugins, slider is a great option to create a stunning gallery of slides containing images, texts, buttons, video etc. There are plenty slider plugins available in the WordPress plugin marketplace. This section will summarize the analysis of the slider plugins that have been tested.

Even though more than 10 plugins have been installed and tested, for the sake brevity, only 5 of them will be discussed, which includes the slider that have been chosen among them to create the Info-Terminal. The five slider plugins are Meta-Slider, MotoPress-Slider, SiteOrigin-Slider, SmartSlider 3, and WD-Slider.

The analysis has taken into consideration aspects such as general properties such as autoplay-capacity and transition effects; available controllers to control the slide which includes play/pause button, next/previous button and thumbnails; the user-interface of the admin page; ease of use in order to create the slider; responsiveness; search engine optimization; customization through CSS and JavaScript; adding layers and layer effects; and load time for 3 slides.

\subsection{Meta-Slider}
Meta Slider is created by Team Updraft and available free of cost. It enables the creation of unique, \ac{seo} optimized slideshow or slider. It is easy to use and has a user-friendly admin page. This include drag-n-drop functionality, setting slide caption and linking the slide etc.

It provides four different types of templates: Flex Slider2, Nivo Slider, Responsive Slider, and Coin Slider. This slider is responsive and support full-width slideshow. The images can be intelligently cropped before adding to slides. Further more the transition effect and speed can be configured through the admin page. The slider is fast, where it takes about 1.23-2.08 seconds to load a slider containing 3 slides.

Even though, it is minimal, lightweight and offers good features as mentioned above, it has not been chosen to build the Info-Terminal due to limited advanced settings to control the slide, no available controllers and unavailability of customization option through CSS and JavaScript.

\subsection{MotoPress Slider}
The MotoPress Slider by MotoPress is a responsive and easy solution to build sliders. It has drag-n-drop features, touch and swipe navigation for touch devices, and fully responsive mobile-friendly layout. This slider creates search engine optimized slider as well. The slider can be built full-width with this plugin.

This is the only slider which support layer effects and styles in the free version. The layer effect is the effect of the elements on the slides such as text, images, button etc. The layer effect can improve the user experience. Other than that, it offer easy configuration and customization through JavaScript and CSS. It also support embedding background videos as well as video from providers such as YouTube and Vimeo.

The admin interface can be said just above average. The load time for 3 slides using MotoPress slider can take up to 6.9 seconds, which is relatively high. Despite of providing extensive slideshow settings (e.g. size, animation, controller, appearance), the slider crashes quite often during the development process. This leads to loss of coding and unsaved work. Due to this, this slider has been opted out.

\subsection{SiteOrigin Slider}
SiteOrigin Slider comes pre-bundled with the SiteOrigin Builder plugin. As the name suggest, it has been developed by SiteOrigin and available free of cost. It provides an easy way to integrate a simple slider into pages that used the SiteOrigin Builder or the widget areas. The slider is responsive and allows unlimited number of slideshows and slides.

The slider provides controller options for animation speed, timeout and navigation color. Other than that, it can be customized by using CSS. The plugin allows adding background image or video in addition to foreground images or videos. The admin page to create the slider is intuitive and can be said above average as for user-friendliness.

The performance measurement for loading 3 slides using this slider plugin lies in between 1.38 to 2.37 seconds. The slider, however, doesn't provide settings to change the controllers, autoplay, size, and animation effects. It also not possible to create a full-width slider with this plugin. Hence, this plugin has not been used to create the Info-Terminal

\subsection{Smart Slider 3}
Smart Slider 3 is another slider plugin created by Nextend. It is a recently released plugin as a successor to Smart Slider 2. It is free slider plugins with many features and ease of use. One of the main advantage of this plugin is the user-friendliness of the admin page. The admin page of this slide is very intuitive, interactive as well as easy to learn. The slider can built full-width.

This plugin offers many features such as slide builder, slide library, and editor. Other than that, it is also fully responsive and mobile-friendly. It offer various settings options such as for the slider size, controllers, autoplay, and animation. These settings can also be customized. Additionally, the slider can be customized through the JavaScript and \ac{css}.

Various elements can be added to the slides including texts, images, buttons, video or our own \ac{html} codes, which enables the building of our own elements. All the elements can be further customized through CSS in addition to the settings such as sizing and positioning provided.

The performance measurement to load 3 slider takes up to 4.21 seconds. Even though, it is not the fastest, considering the amount of customization it allows, the slider has been chosen to build the Info-Terminal. Further more, in the paid version of this slider plugin, the layer effect can be added to slider elements. This feature can be considered for the future development.

\subsection{WD Slider}
WD Slider by WedDorado is yet another slider plugin that has been tested and considered for the Info-Terminal. The slider created through this plugin can be added into the web page or as widget. It supports images and videos.

In addition to that, it has a lot options to control the slide show such as slide transition, slide timing, autoplay, aligning and image cropping. The plugin has features such adding watermark, music. It is responsive and allows full-width. Developers can customize the slides through CSS and JavaScript.

The admin panel of this plugin is moderately friendly. The free version has limited scope despite of allowing unlimited numbers of slideshows and slides. The plugin also doesn't support layer effect for animation. The loading time for 3 slides is quite high as measured, 5.18 seconds. For these reasons, it has not been chosen to build the Info-Terminal.