\chapter{Background Works}
The background works involved in this thesis in order to create the Smart Home Lab website divided into two parts. The first part is the analysis of selected Smart Home Lab websites of other institutions. The second part is the analysis of the WordPress plugin in order to create the Info-Terminal.

\section{Website Analysis}
In this work, altogether six websites have been analyzed. These include FZI Forschungszentrum, KIT iZEUS, IoTLab Reutlingen, Duke Smart Home Program, MIT Mobile, and MIT Smart Living. In general, criteria such as the layout, contents, main navigation menu, media supported, responsiveness and languages supported have been analyzed.

\subsection{FZI Forschungszentrum}

\subsubsection*{Website Information}
\begin{itemize}
\item URL: https://www.fzi.de/en/home/
\item Web server: Apache
\item Application server: PHP 5.4.45
\item CMS: Typo3
\end{itemize}

\subsubsection*{Web Layout}
The website uses normal layout, not screen-wide. The theme look modern, but the normally used one. The Header contains institution logo on the right, and the top part has option to change language, link to contact page and search bar. Lower part of the header has main navigation menu (Home, News, Research, Our Offer, Work For Us, and About Us). The header is fixed, but collapse when user scroll down.

Body is two column sized. In all pages except home page, the small left column has sub menu respective to that individual page and large right column has the content. On the main page, the the left column is relatively bigger which has slideshow and latest news feed. Right column is relatively small and has list of upcoming events and quicklinks.

Footer is small sized. It has sub menu on the left (Home, Privacy, Legal Notice, Sitemap, Search) It also has social media plugins on the (Xing), RSS feeds and contact links.

\subsubsection*{Contents}
Contents presented on the website are the home page (featured article/news, upcoming events, latest news, quicklinks), newsfeed, research (research sector, research focuses, projects etc), offers, work for us (carrier page), about us, privacy, legal notice, and sitemap. As analyzed, the website has a lot of contents, perhaps it could have been organized in a better way.

\subsubsection*{Menu}
There are 3 types of menu:
\begin{itemize}
\item Main navigation menu situated at the header (Home, News, Research, Our Offer, Work For Us, and About Us)
\item Sub menu respective to individual pages which is different from one another located at right column of the body part
\item Footer menu located the footer (Home, Privacy, Legal Notice, Sitemap, Search).
\end{itemize}

\subsubsection*{Media}
The website supports the following medias, such as images, slide show, videos, contact form, search bar, social media button.

\subsubsection*{Responsiveness}
The webpage is responsive. The header is compressed. The main navigation menu is shifted to the bottom of the page with a link to the menu located at the header. The two column body part is changed to single column on window resize. The sub menu on the individual pages are merged with main navigation menu in hierarchical order, which is situated at the bottom of the page.

\subsubsection*{Languages}
The website supports German and English languages.

\subsection{KIT iZEUS}
\subsubsection*{Website Information}
\begin{itemize}
\item Url: http://www.izeus.kit.edu/62.php
\item Web server: Apache 2.4.10
\item Application server: PHP
\item Server: Debian
\end{itemize}

\subsubsection*{Web Layout}
The web layout is classic normal one. It is not wide and not stretched. The institution logo is located in the header at the top. The sub-menu can be spotted small in the upper right part of the header. The main frame is divided into three columns. The left column contain the main navigation menu. The center column serves as the main content. Lastly, the right column fixed banner. The footer is small and has copyright statement.

The web layout is classic without fancy popup animations, graphics, slider etc. It has to also noted that it is most common and user friendly where most users know how to navigate.

\subsubsection*{Contents}
Contents presented on the website are the home page, energy smart home lab, information materials, project consortium (partners and KIT chairs), publications, press review, links, contact, legals and sitemap. The content look organized, clear to be viewed and read. Navigating through the contents is also easy, but it appears not to be updated anymore.

\subsubsection*{Menu}
The website contains two menus:
\begin{itemize}
\item Small top right menu (Home, Lang Pref, Legals, Sitemap, Link to KIT)
\item Main Menu on Left Column (Home, Energy Smart Home Lab, Information Material, Project Consortium, Publications, Press Review, Links, Contact)
\end{itemize}

Analysis: Both navigation are visible on all pages, static, easy to use, simple hovering effect, highlight on active link.

\subsubsection*{Media}
The website supports the following media, video, images, banners, PDFs, and external links. However, no animation, picture gallery and image slider can be spotted.

\subsubsection*{Responsiveness}
The website is not responsive. It is best viewed with desktop and laptop. Having said that, it is not mobile- or tablet-friendly and not recommended for Microsoft Surface Hub.

\subsubsection*{Languages}
The website has German as the primary language and supports English as an alternative language. The website is designed will on both the languages.

\subsection{IoTLab Reutlingen}
\subsubsection*{Website Information}
\begin{itemize}
\item URL: http://iotlab.reutlingen-university.de
\item Web server: Apache
\item Application server: PHP 5.5.36
\item CMS: Joomla!
\end{itemize}

\subsubsection*{Web Layout}
The website uses a modern web layout and the web page is full-width stretched. The header contains brand logo and horizontal main navigation menu. The header also contains toggle button to open / close main navigation menu. It is fixed and the both menus are duplicated.

The website is created using Joomla! content management system. The home page is single columned with newsfeed. Whereby, the other pages are double column, with right column containing sub-menu and left column showing the main content. Other that that, fixed social media icons (e.g. facebook and twitter) can be found on right edge of the website.

The footer contains copyright statement, impressum link, link to university homepage, link and info about CMS, and link / info about CMS theme. The web layout doesn't look systematic and well organized even though 
it uses modern web templating. The users might face difficulty to get  information or contents they are looking for.

\subsubsection*{Contents}
The website contains the contents such as new feeds on home page, master projects listing (current and finished master projects, team , introduction), publications, thesis, blogs (containing 3 categories: IoTLab-Blog, Mobile Computing and Distributed), systems, contact form, and commenting for blog article.

\subsubsection*{Menu}
The website has three menus:
\begin{itemize}
\item One menu located on the webpage header (Home, Master Project, Publications,  Thesis, Blog), but Hidden on mobile / tablet mode.
\item Sidebar Menu which has to be toggle opened/closed using button (Duplicate of  the main menu located on the header with additional menu "Intern")
\item Menu located at left column of pages (Duplicate of the menu with some  additional menus such as Cooperation, News, Events etc.)
\end{itemize}

The menus are duplicated which is inconvenient and confusing for the user to navigate around.

\subsubsection*{Media}
The website contains images, slide shows, and video. It also supports PDFs, banner and animation.

\subsubsection*{Responsiveness}
The website is responsive and adapts to various screen sizes. The contents are readable and viewable using different devices. It is mobile- and tablet-friendly.

\subsubsection*{Languages}
The website supports only English language.

