\chapter{Conclusion}
This thesis work involves creating a website for the smart home laboratory at the Hochschule Furtwangen University. The website has been designed and developed by using content management system called WordPress. WordPress is an open source CMS which is widely used and has a lot features built-in.

In addition to that, the website has been equipped with an info-terminal. The info-terminal serves to display information regarding the laboratory and its components in an automated and interactive way. Hence, a WordPress plugin called SmartSlider 3 has been used to develop the info-terminal. This plugin is also open-source and can be downloaded from WordPress plugin marketplace.

Before choosing the plugin SmartSlider 3, the thorough research and analysis have been carried out. This research involves learning available methods and plugins which will allow the development of the info-terminal inside the WordPress. Since WordPress doesn't offer much of a flexibility like programming on native platform offers, the result that can be achieved through the chosen plugin has to be maximized. More 10 different plugins has been downloaded, installed and tested on a local server. From the analysis, SmartSlider 3 has been found to be the most appropriate plugin to be used for this purpose.

The website also has been developed to include the live stream of an IP camera footage. In order achieve this, the IP camera has been configured and the parameters required to connect and communicate with it has been studied. Then, a web page has been designed and programmed in the WordPress to embed the live footage of the IP camera in real-time. Additionally, buttons has been added and programmed, which enables controlling the IP camera remotely.

Another task that has been accomplished through this thesis is the integration of Unity 3D-Model. This task accounts to only a small portion of the thesis work. The task involves only in uploading necessary files into the server and developing a web page to render and display the 3D-Model correctly.

The safety of the website has been ensured from the server level up to the front-end level. Then, it has been public which can be accessible every one.

\section{Future Works}
Even though the website is made public and fully functional, there are still room for improvements. For an example, more contents can be added to the website. Other than that, the blogging features has been installed in the WordPress. It only requires activation as discussed in the previous chapter, and latest news regarding the development in the smart home laboratory can be shared instantly.

For the info-terminal, the interactivity can be enhanced through the CSS3 animation. The CSS3 animation allows the HTML elements displayed on a web page to be animated through various effects. There are two ways to this. First is to program the animation manually and applying to the elements such as images and texts on the info-terminal. Secondly, the plugin SmartSlider 3 full version can be bought. By upgrading to full version, the animation can be added through the plugin user interface without any programming works.

