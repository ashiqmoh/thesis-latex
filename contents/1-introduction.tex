\chapter{Introduction}

The Smart Home Laboratory has been established at Hochschule Furtwangen University, Campus Furtwangen a few years ago. This laboratory can be found at B2.01. Currently, this laboratory does not have an official website yet. Hence, through this master thesis, a website has to be created by using the \ac{cms}, called WordPress.

This website allows the university's professors and students, as well as public to access and learn about the smart home lab interactively through any web browsers. In order to achieve that, the website will be equipped with an Info-Terminal, which will display all the available systems, devices, as well as sensors and actors in the lab.

Here is a description of Info-Terminal. The Info-Terminal is an automated tour i.e. endless presentation. It will provide information regarding the components, panels and use cases in the Smart Home Laboratory. Additionally, it has controllers such as play, pause, next, previous and home buttons. Other than than, there are thumbnails at the bottom to navigate and view all the slides available in the info terminal. This will allow the users to have an overview of the smart home laboratory and navigate it interactively.

Other than that, a live stream of CCTV recording has been programmed and added to the website. This CCTV can be found in the lab. The website enables the privileged users to view live recording through the website. In addition to that, a 3D-Model made through the Unity technology has been integrated into the WordPress.

The Info-Terminal has to be programmed as such that it can be viewed through a touch display called Microsoft Surface Hub, which can be found in the lab. Though the website as well as the Info-Terminal has been programmed and designed to be viewable across all devices, it has been a little more consideration to be best suited on the Microsoft Surface Hub.

\section{Aim and Objectives}
The main aim of this master thesis is to create a website using WordPress, which contains Info-Terminal and displays live CCTV recordings.

Here is an overview of objectives that has been achieved in this thesis work:
\begin{itemize*}
\item Preparation of expose, documentation structure, literature sources, time-planning and calendar-planning
\item Research on smart home lab websites and contents of other universities
\item Server setup and installation of WordPress
\item Tutorials on WordPress, its theme and plugins
\item Structuring, designing and adding the contents to the website
\item Building the Info-Terminal based on the systems and devices available in the lab.
\item Connecting CCTV recording to the website
\item Integrating 3D Unity model
\item Setting up email server
\item Ensuring security and safety of the server and the website
\end{itemize*}

\section{Project Management}
The thesis has to be done in six months period. It has been started on the first of March and has to completed and submitted by the end of August 2017. This includes finishing up the practical part and the documentation on it.

The thesis is started officially with creating expose and defining aims as well as constructive objectives. At the beginning stage, the project management part for this thesis has been planned. As planned, status reports and thesis planning has to be submitted every two weeks on the odd calendar week. Other than that, meeting and discussions have been held every four weeks to discuss on the outcome and further works that has to be carried out.

In the project management phase, the time planning and calendar planning has been organized accordingly with the tasks that has to be carried out through out six months time period. Before the implementation of website has been started, context diagrams has been created as a summarization of the practical tasks that were to be done. Apart from that, the contents for the documentation has been structured to mediate the report writing at the later stage.

\section{Documentation}
This documentation consist of following chapters:
\begin{description}
\item[Chapter 2] discusses the related background works and researches
\item[Chapter 3] outlines the ideas and approaches involved in solving the problem
\item[Chapter 4] discusses the technologies used in order to create the website
\item[Chapter 5] explains the server setup and installation of required software
\item[Chapter 6] shows the important steps involved in designing the website
\item[Chapter 7] shows steps involved in creating Info-Terminal
\item[Chapter 8] outlines on work involved in streaming CCTV live recording
\item[Chapter 9] shows how to integrate 3D unity model into a WordPress powered website
\item[Chapter 10] explains step-by-step on how to enable blogging on the website
\item[Chapter 11] summarizes this thesis work with future outlook
\end{description}